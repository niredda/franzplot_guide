\documentclass{article}
\usepackage{hyperref}
\usepackage{amsmath}
\usepackage{amssymb}
\usepackage{pgf}
\usepackage{tikz}
\usepackage[top=0.5in, bottom=0.75in, left=0.5in, right=0.5in]{geometry}
\usetikzlibrary{matrix}
\newcommand{\fig}{figures} % common figure path
\newcommand{\frnzplt}{FranzPlot }
\title{Franzplot-DCS}
\author{F. C., D. R., A. S.}
\begin{document}
\maketitle
\tableofcontents
\newpage
\section{Introduction}\label{s:intro}
\section{Technical aspects}\label{s:tech_asp}
\section{The top bar}\label{s:top_bar}


\begin{tikzpicture}
    \node[anchor=south west,inner sep=0] (image) at (0,0) {\includegraphics[width=0.9\textwidth]{\fig/topbar.jpeg}};
    \begin{scope}[x={(image.south east)},y={(image.north west)}]
        \draw[red,ultra thick,rounded corners] (0.62,0.65) rectangle (0.78,0.75);
        %\node[] (lab1) at (0.62,0.65) {1};
    \end{scope}
\end{tikzpicture}
\begin{figure}
\includegraphics[width=\textwidth]{\fig/topbar.jpeg}
\caption{\label{fig:topbar}}
\end{figure}
In attesa di un termine pi\'u adatto
\section{The top menu}\label{s:top_menu}
\begin{figure}
\includegraphics[width=0.4\textwidth]{\fig/topmenu.jpeg}
\caption{\label{fig:topmenu}}
\end{figure}
Come sopra
\section{The graph menu}\label{s:graph_menu}
\subsection{Geometries}\label{ss:geometries}
\subsubsection{Curve}\label{sss:curve}
\subsubsection{Bezier curve}\label{sss:bez_curve}
\subsubsection{Surface}\label{sss:surface}
\subsubsection{Plane}\label{sss:plane}
\subsubsection{Primitive}\label{sss:primitive}
\subsection{Parameters}\label{ss:parameters}
\subsubsection{Interval}\label{sss:interval}
\subsubsection{Value}\label{sss:value}
\subsubsection{Sample parameter}\label{sss:samp_par}
\subsection{Transformations}\label{ss:transformations}
\subsubsection{Generic matrix}\label{sss:gen_mat}
\subsubsection{Rotation matrix}\label{sss:rot_mat}
\subsubsection{Translation matrix}\label{sss:transl_mat}
\subsubsection{Transform}\label{sss:trans}
\subsubsection{Time transform}\label{sss:time_trans}
\subsection{Point}\label{ss:point}
\subsection{Vector}\label{ss:vector}
\subsection{Geometry renderer}\label{ss:geo_rend}
\subsection{Vector renderer}\label{ss:vec_rend}
\section{Examples}\label{s:examples}
\section{Trouble Shooting}\label{s:trouble_shooting}
\end{document}
